\documentclass[11pt, oneside]{article}   	% use "amsart" instead of "article" for AMSLaTeX format
\usepackage{geometry}                		% See geometry.pdf to learn the layout options. There are lots.
\geometry{letterpaper}                   		% ... or a4paper or a5paper or ... 
%\geometry{landscape}                		% Activate for rotated page geometry
\usepackage[parfill]{parskip}    		% Activate to begin paragraphs with an empty line rather than an indent
\usepackage{graphicx}				% Use pdf, png, jpg, or eps§ with pdflatex; use eps in DVI mode
								% TeX will automatically convert eps --> pdf in pdflatex		
\usepackage{amssymb}
\usepackage{color}

%SetFonts

%SetFonts


\title{Tap thingy}
\author{Me}
%\date{}							% Activate to display a given date or no date

\begin{document}
\maketitle

Hello this is a very exciting report which will hopefully be more exciting by the end of December \cite{Weltmann2009}. And the references work \cite{Schroter2015atomic}. Wooo.

\section{Background}

\subsection{Wounds}

Wounds and their management are an issue both in the UK and globally \cite{Posnett2008burden}.
Methods of accelerating wound healing to decrease infection risk are sought, especially in developing countries where wound infection rates are extremely high \cite{Kihla2014risk}. 
Common wound-infecting pathogens such as \textit{Staphylococcus aureus} and \textit{pseudomonas aeruginosa} \cite{Church2006burn, Bowler2001wound} are becoming increasingly resistant to antibiotics, therefore, new wound treatments appropriate for the healthcare budgets of both developed and developing countries are required \cite{Chambers2009waves, Godebo2013multidrug, Howell2005a}.
One technique under investigation is low temperature plasma (LTP), which has shown promise for both bacterial killing and wound healing promotion \cite{Kong2009plasma, Kramer2013suitability, Isbary2012successful, Isbary2010a}.


\subsection*{Low Temperature Plasma}

Plasma is quasineutral ionised gas, considered to be the fourth state of matter \cite{Fridman2013plasmamedicine}.
Low temperature plasma (LTP), specifically, is formed when only a small percentage of gas particles are ionised, resulting in a low electron density and an overall low plasma temperature (approximately room temperature).
The appealing properties of LTP arise due to electron-mediated processes.
Firstly, dissociation of molecules by electron impact produce, for example, free radicals, including reactive oxygen and nitrogen species (RONS), which are known to be bactericidal \cite{Kong2009plasma}.
Secondly, electronic excitation from electron impact causes excitation of molecules and subsequent emission of, for example, UV radiation which is known to be bactericidal at certain wavelengths and powers \cite{Laroussi2004evaluation}.
Thirdly, sufficiently energetic free electrons can cause ionisation of other atoms/molecules to sustain the plasma and produce ions that can contribute to the bactericidal effects of LTP \cite{Mendis2000a, Laroussi2002nonthermal}.
LTP has shown significant promise in the eradication and/or inactivation of many microorganisms, including those in biofilms \cite{Laroussi2005low}. 
There is also evidence to show that LTP may promote the wound healing process in eukaryotic cells \cite{Haertel2014nonthermal, Kramer2013suitability}.
Identification of specific roles of different plasma components could help with tailoring plasmas for different biomedical applications.

Here, we propose the development of an LTP device utilising an air feed gas, which will be characterised both experimentally and computationally, and its potential for use in wound infection and healing applications assessed through various \textit{in vitro} biological assays.
The investigation into using an air feed gas is beneficial when considering low-cost wound treatments, as it should reduce running costs, and increase device portability, through eliminating the need for bottled gases.

\section{Modelling}
Modelling of plasmas in general is useful to be able to compare to experimental data. 
It is then useful to be able to indicate which parameters may be the most influential for altering the concentrations of different species which may be particularly useful in terms of biological activity.

\subsection{GlobalKin}
GlobalKin is a 0 dimensional global chemistry plasma model which has three main parts - A reaction chemistry and transport module, a Boltzmann equation solver for determining electron energy distributions (EED) and an ordinary differential equation (ODE) solver \cite{Stafford2004O2}.
The chemistry and transport module constructs differential equations for the evolution of species concentrations and temperatures over time. 
This takes into account the electron impact rate coefficients, which are calculated as a function of the electron energy by the Boltzmann solver.
The differential equations are then solved by the ODE solver.

The time evolution of species densities is determined using the following equation:

\begin{equation}
\frac{dn_i}{dt} = \frac{1}{\Lambda_D^2}\bigg(-D_iN_i + \sum_jD_jN_j\gamma_jf_{ji}\bigg) + S_i - \frac{N_i}{T_g}\frac{dT_g}{dt}
\label{SpeciesEvolutionEqn}
\end{equation}

where $N$ is the number density of heavy species $i$, $\Lambda_D$ is the diffusion distance, $D$ is the diffusion coefficient, $\gamma$ is the sticking coefficient of the species at the walls, $f_{ji}$ is the fraction of species $j$ that returns from the wall as species $i$, $S_i$ is the source term and $T_g$ is the gas temperature.

The first term of equation \ref{SpeciesEvolutionEqn} refers to the interactions of species with the walls, the second is a source term for species, which takes into account the reactions producing and consuming it and the third term relates to the temperature.

The source term is calculated using the following equation:

\begin{equation}
S_i = \sum_j(a_{ij}^{RHS}-a_{ij}^{LHS})k_j\prod_lN_l^{a_{ij}^{(LHS)}}
\label{SourceTermEqn}
\end{equation}

%Sigma is summation (as in add everything up) whereas capital Pi is where you do the product of everything (multiply) 
where $a$ is the stoichiometric coefficients of species $i$ in reaction $j$ on the right and left hand side (RHS and LHS, respectively) of the reaction equation, $k$ is the reaction rate coefficient for reaction $j$.
\textcolor{red}{ $\prod_lN_l^{a_{ij}^{(LHS)}}$ is something to do with electron processes which also contribute to the species densities. 
In the original thesis \cite{Dorai2002modeling}, the last term is $\prod_lN_l^{a_{lj}^{(LHS)}}$, whereas in \cite{Stafford2004O2} it is $\prod_lN_l^{a_{ij}^{(LHS)}}$. 
Not sure which is correct, but presumably the thesis as $l$ is referring to electron processes.}
The reaction rates for use in equation \ref{SourceTermEqn} are calculated using Arrhenius expressions for heavy species only reactions. 
However, for electron-impact reactions, these are calculated as a function of the electron energy.
Therefore, in the reaction chemistry set, only an electron-impact cross section is specified and the reaction rate is then calculated using the electron energy distribution function (EEDF) calculated internally by the GlobalKin Boltzmann equation solver.

As well as consideration of species, the electron temperature is also calculated by GlobalKin, taking into account power deposition in the plasma and electron energy losses through elastic and inelastic collisions as follows:

\begin{equation}
\frac{d}{dt}\Big(\frac{3}{2}k_BT_e\Big) = P_d - \sum_i\frac{3}{2}n_e\nu_{mi}\Big(\frac{2m_e}{M_i}\Big)k_B(T_e - T_i) + \sum_l n_ek_lN_l\Delta\epsilon_l
\label{ElectronTempEqn}
\end{equation}

where $n_e$ is the electron density, $T_e$ is the electron temperature, $P_d$ is the plasma power input, $m_e$ and $M_i$ are the masses of electrons and heavy particles, respectively, $\nu_{mi}$ is the collision frequency, $k$ is the reaction rate, $N_l$ is the gas phase collision partner density (presumably for electron collisions/processes?) and $\Delta\epsilon_l$ is the electron energy loss.

\subsection{GlobalKin Inputs}
There are a number of inputs required for GlobalKin to work.
Firstly plasma parameters are needed.
These include the plasma volume, power input, gas feed composition and volume, gas and wall temperature, plasma surface area, diffusion length and initial molar fractions of each species.
As well as these plasma parameters, it is also specified how regularly the EEDF will be updated.

Secondly, the reaction chemistry set has to be specified. 
This has two parts, (1) a list of species in the plasma as associated parameters and (2) a list of all the possible reactions in the plasma and their associated parameters described below.

\subsubsection{Species}
Every species that is included in the model has to be specified with its enthalpy of formation, charge and molecular weight.
As well as this, other parameters are required, such as Lennard Jones parameters (related to the potential between two atoms/molecules) and the characteristics of species interaction with walls.
%Lennard Jones parameters 1 and 2. 1 is $\sigma$ which is the internuclear distance between atoms where the potential between them is zero. Parameter 2...? Units are kelvin?!
In particular:
\begin{itemize}
\item the sticking/disappearance coefficient is the fraction of the species that will be lost through interaction with the wall ($\gamma$ in equation \ref{SpeciesEvolutionEqn})
\item the return branching fraction is the fraction of the species of interest returning to the plasma following a different species being lost to the walls
\item the return species is the species that this species will produce if it is lost to the wall
\end{itemize}


\subsubsection{Reactions}
Every reaction taking place in the plasma also has to be specified along with a set of parameters.
These parameters are basically the things required to calculate the reaction rate coefficient for each reaction.
The methods for doing this are generally different for heavy species reactions and electron reactions, due to the strong dependence of reaction rate on energy.
For heavy species reactions, a specific rate coefficient can be specified in terms of Arrhenius equation coefficients, as ion energies do not change much due to their heavy mass.
However, for light electrons that are able to easily gain energy from the electric field as well as collide and lose energy, their energies can vary greatly (as shown by the EEDF), therefore, a constant rate coefficient would not be appropriate. 
Instead a reaction cross section is specified and the rate constant can then be calculated for electron processes by combining this cross section with the internally calculated EEDF.

Alongside this, other things are specified, such as change in enthalpy during the reaction which contributes to gas heating and whether or not the collision is superelastic.

%\begin{itemize}
%\item Arrhenius equation coefficients. Usually electron reaction rates don't use Arrhenius equation coefficients and instead use cross section and internal EEDF.
%\item Activation energy
%\item Reaction cross section/special number?
%\item Whether reaction can be superelastic
%\item Change in enthalpy in reaction which appears as gas heating
%\item Electron energy loss in collisions
%\end{itemize}

\subsection{GlobalKin Outputs}
After running, GlobalKin, produces output files which list each species density at each small time step in the full simulation time.
Using the gas flow velocity, each time step can be translated into a position along a plasma channel and can therefore, give an indication of spatial distribution along the plasma channel, even though the code does not take into account any spatial dimensions.

Other outputs also include gas temperature.....






\section{Examples of using GlobalKin}
GlobalKin can be used to simulate experimental conditions so that simulations and measured data can be compared.
The data below shows comparisons between measured and simulated hydroxyl radical densities using different plasma parameters.

The measured data is from absorption spectroscopy, using an atmospheric pressure radio-frequency plasma source with a gas feed of He with admixtures of H\textsubscript{2}O (obtained by diverting some of the He feed gas through a bubbler to saturate with water vapour).

\subsection{Spatial Distribution}

\subsection{Power Variation}


\subsection{European Collaboration}
A collaboration was recently started across many plasma groups in Europe, with the aim of developing a global plasma model for air plasmas.
This is of interest in my project, and provides a good opportunity to continue with modelling and be involved in such as initiative.

The idea is that each research group uses their own plasma model and chemistry set, then specific outputs from each group can be anonymised and distributed through the entire consortium.
Following this, there can be iterations of the process until groups are happy with their simulations/outputs.
The collaboration will be run in a series of stages.
Firstly, codes will all be initialised, using parameters and short reaction sets provided by the coordinator, so that outcomes can be compared and groups can check that their codes are appropriate for the task.
At this point, we need to make sure that GlobalKin is a suitable model and able to carry out simulations using the specified inputs in order to give the correct outputs.
The deadline for this stage is currently 20.1.17.
Following this, the simulations of air plasmas can begin.

As part of this collaboration I am investigating air reaction chemistry sets that have been used before for air plasma modelling, and I will be carrying out simulations.


\section{Modelling Air Plasmas}
\subsection{What's been done before?} 
There are many different models and chemistry sets that have been used for previous attempts at modelling air plasmas. 
Issues with modelling atmospheric pressure air plasmas is the number of species that are involved and the high collisionality environment.
\subsection{Chemistry sets}
For the purposes of both the collaboration, and my own research project, the aim is to find an appropriate air chemistry set that can be used for modelling.
Firstly, in the original thesis and subsequent publication presenting the GlobalKin code, there is a full reaction chemistry set presented for modelling humid air plasmas interacting with polypropylene surfaces \cite{Dorai2002modeling, Dorai2003a}.
Here, they consider two sets.
Firstly the gas phase (consisting of species formed from, and reactions involving the flow gas containing N$_2$/O$_2$/H$_2$O in 79/20/1 proportions), and secondly the surface interaction chemistry. 
For the purposes of our model, only the gas phase reactions are of interest.
The gas phase consists $\sim$334 reactions and 56 species. 
The model used is GlobalKin and, therefore, works as outlined above.

However, there are other types of models that have been used for modelling air plasmas.
For example, in a recent paper by Kutasi $et al$ \cite{Kutasi2016tuning} they present a model for an Ar/N$_2$/O$_2$ plasmas.
Their main aim is to investigate the afterglow region of a plasma, both using Ar/N$_2$/O$_2$ mixture and using just N$_2$/O$_2$.
The simulation works by solving the homogeneous electron Boltzmann equation, alongside the rate balance equations for the creation and loss of important heavy particles (ions and neutrals). 
The maintenance reduced electric field is also calculated self-consistently.

Unfortunately, the full reaction set is not presented, however, the N$_2$-O$_2$ reactions are taken mainly from \cite{Guerra1997self, Pintassilgo2005modelling, Kutasi2008modelling}.
In 1997, Guerra $et al$ \cite{Guerra1997self} presented a paper (as an extension to a previous publication \cite{Guerra1995non}), whereby 


%is a paper building on a previous one \cite{Guerra1995non}, where some vibrational states of N$_2$ had to be an input parameter of the model, taken from experimental data. Here, the model consists of the electron Boltzmann equation coupled to rate balance equations for vibrationally excited molecules of N$_2$ and O$_2$, electronically excited states of N$_2$ and NO, N and O species (with term symbols attached?). All this plus continuity equations for electrons and main positive ions (N$_2^+$, N$_4^+$, O$^+$, O$_2^+$, NO$^+$). TO DO - read this paper to find out the point of what it was doing. What was the model doing? 
%\cite{Pintassilgo2005modelling} is looking at sterilisation processes and therefore using a model to predict ways of maximising concentrations of NO ($B ^2\Sigma$) and O($^3P$). 
%The model starts by working out the EEDF and also the vibrational distribution function of some molecules. 
%It also calculates the concentrations of N$_2$ and O$_2$ electronic states, N and O atoms, NO, NO$_2$ and O$_3$ species, as well as the positive and negative ions formed in the discharge. However, not sure what the type of model is because very few reactions shown...
%\cite{Kutasi2008modelling} is a paper about a 3-D hydrodynamic model, with regards to UV emission from the plasma, to be used for sterilisation.
%
%However, whilst they do not cite their entire reaction chemistry, they do provide references for reactions involving the different gases. Of particular interest are the references relating to reactions involving N$_2$ and O$_2$.
%
%There are lots of chemistry sets that have been used before for air plasma modelling.
%Spacecraft re-entry? 
%\begin{itemize}
%\item Lazarou2016numerical - He/Air plasma. Investigating the effect of air in He plasmas and therefore the effects of impurities on the running of He plasmas. 27 species, 153 reactions. No NO$_x$ species included as increased computational time too much without affecting simulation results significantly \cite{Lazarou2016numerical}.
%\item Murakami2014afterglow -  This is modelling air impurities in He/O$_2$ plasmas too. Very small percentage air mixture ($\approx$ 0.025\%). The importance of the air in the model was to see how different percentages of air impurities in the He plasma affected the densities of different plasma species.\cite{Murakami2014afterglow}
%\item here is another paper \cite{Gordiets1995kinetic}.
%\item Rate constants for reactions in global chemistry models affect species density evolution over time. 
%By evaluating the reactions/rate constants which contribute the most to a He/O$_2$ plasma chemistry, the comprehensive reaction scheme (25 species, 373 reactions \cite{Turner2015uncertainty}) was rationalised to a reduced scheme (12 species, 51 reactions). The initial rate constants determined through experimentation/theory have an associated error with them. Therefore, the study also looked at which rate constants for each of the included reactions contributed the most error to the system. It was found that it was a small proportion of the reaction rate constants that were responsible for the majority of the error. In particular, rate constants for 3 body reactions involving He and electron reactions with O species, had the highest contribution to the overall error \cite{Turner2016uncertainty}.
%\item A collisional radiative model (looking at the distribution of atoms/molecules over their excited states \cite{Sijde1984collisional}) was developed to look at air plasmas formed during spacecraft entry to upper levels of a planets atmosphere. This isn't really what I'm looking for though as it is too much to do with different vibrationally excited states... I think \cite{Bultel2006collisional}.
%\item Really useful paper that used GlobalKin for argon plasma going into humid air, but also has good intro on some air plasma chemistry sets \cite{Gaens2013kinetic}.
%\item This paper talks about modelling surface microdischarges. Ie from top to bottom: driven electrode, dielectric, discharge, grounded mesh, effluent, surface to be treated. 53 species, 624 reactions \cite{Sakiyama2012plasma}.
%\item Kutasi \cite{Kutasi2016tuning}(Vasco paper) does not cite it's full chemistry set for Ar/O$_2$/N$_2$. However, it says it's set it based mainly on other papers. Of particular interest it's N$_2$/O$_2$ interactions come mainly from \cite{Pintassilgo2005modelling, Guerra1997self, Kutasi2008modelling}.
%\item \cite{Guerra1997self} is a paper building on a previous one \cite{Guerra1995non}, where some vibrational states of N$_2$ had to be an input parameter of the model, taken from experimental data. Here, the model consists of the electron Boltzmann equation coupled to rate balance equations for vibrationally excited molecules of N$_2$ and O$_2$, electronically excited states of N$_2$ and NO, N and O species (with term symbols attached?). All this plus continuity equations for electrons and main positive ions (N$_2^+$, N$_4^+$, O$^+$, O$_2^+$, NO$^+$). TO DO - read this paper to find out the point of what it was doing. What was the model doing? 
%\item \cite{Pintassilgo2005modelling} is looking at sterilisation processes and therefore using a model to predict ways of maximising concentrations of NO ($B ^2\Sigma$) and O($^3P$). 
%The model starts by working out the EEDF and also the vibrational distribution function of some molecules. 
%It also calculates the concentrations of N$_2$ and O$_2$ electronic states, N and O atoms, NO, NO$_2$ and O$_3$ species, as well as the positive and negative ions formed in the discharge. However, not sure what the type of model is because very few reactions shown...
%\item \cite{Kutasi2008modelling} is a paper about a 3-D hydrodynamic model, with regards to UV emission from the plasma, to be used for sterilisation.
%
%\item \cite{Dorai2002modeling} is a thesis from the Kushner group who designed GlobalKin. 
%In this, a full reaction set is listed for humid air interaction with polypropylene surfaces.
%It says there are 90 species and nearly 400 reactions included in the set (56 species containing only N, O and H. Lots of reactions).
%
%
%\end{itemize}
%
%\section{Electron orbitals, angular momentum and spin}
%Understanding term symbols:
%\begin{equation}
%^{2S+1}L_{J}
%\end{equation}
%where $S$ is the total spin quantum number, $L$ is the orbital quantum number (i.e. S, P, D, F etc), and $J$ is the total angular momentum quantum number.
%$S$ is related to the number of unpaired electrons present in the outermost electron shell.
%Each electron $S = \pm \frac{1}{2}$, therefore, by adding up the spin of every electron, this gives the overall spin. If there are no unpaired electrons, $S = 0$. However, if there is one, spin-up free electron, $S = +\frac{1}{2}$ etc. 
%The term $2S + 1$ gives the multiplicity of the atom/molecule ($2S + 1 = 1 \rightarrow singlet, 2 \rightarrow doublet$ etc)
%Therefore, for the above atom where $S = +\frac{1}{2}$, $2S + 1 = 2$, making it a doublet.
%L is given by the type of outer shell, i.e. S, P, D, F etc orbital.
%
%
%\section{Experimental}
%\subsection{What's been done}
%Nothing
%\subsection{What are the next steps}
%Aim to measure NO? And see what happens to cells when treated with the plasma?!
%Combine what we see with model... May be He/Air (He/O$_2$/N$_2$) gas mixture.
%

\bibliographystyle{ieeetr}
\bibliography{/Users/hld523/Bibliography/MyPapers}
\end{document}  